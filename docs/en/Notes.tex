\documentclass{article}
\usepackage[utf8]{inputenc}
\usepackage[T1]{fontenc}
\usepackage{geometry}
\usepackage{enumitem}
\usepackage{graphicx}

\geometry{margin=0.7in}

\begin{document}

\title{\textbf{Notes}
\\ \large{\textit{Ganics}}}
\author{\textbf{Adrian Chmiel}}
\date{\today}
\maketitle

\section{Introduction}
This file contains various notes that helped me learn more about how different Generative Adversarial Networks work and what particular layers do. It is a summary of the most important information I found during the research.

\section{Generative Adversarial Networks}
\subsection{Basic GAN}
\begin{itemize}
    \item one generator and one discriminator
    \item training on paired data aiming to generate realistic general images
\end{itemize}

\subsection{CycleGAN}
\begin{itemize}
    \item two generators and two discriminators
    \item training on unpaired data aiming for image style conversion
    \item \textbf{cycle consistency loss} - an image translated from A to B and then back from B to A should be similar to the original image
\end{itemize}

\subsection{SRGAN}
\begin{itemize}
    \item one generator and one discriminator
    \item training aiming to generate high-quality images
    \item \textbf{residual connections} - passing information from previous layers to subsequent layers bypassing intermediate layers
\end{itemize}

\subsection{PatchGAN Discriminator}
\begin{itemize}
    \item discriminator evaluating images at the patch level
    \item instead of evaluating the entire image, it evaluates individual image patches
    \item allows for a more detailed evaluation of images
    \item often used as part of \textit{SRGANs}
\end{itemize}
\newpage

\section{Layers \textit{tf.keras}}
\subsection{Sequential}
\begin{itemize}
    \item allows for creating layers in the order they are added
    \item a simple way to build models in \textit{tf.keras}
    \item for building models where layers are applied one after the other in a linear manner
\end{itemize}

\subsection{Concatenate}
\begin{itemize}
    \item concatenates a list of tensors along a specified axis
    \item for combining outputs from different layers, features from earlier layers are combined with features from later layers
\end{itemize}

\subsection{Conv2D}
\begin{itemize}
    \item 2D convolutional layer, used for processing image data
    \item applies convolutional filters that move across input data to extract features
    \item for feature extraction from images, such as edges, textures, etc.
\end{itemize}

\subsection{Conv2DTranspose}
\begin{itemize}
    \item inverse 2D convolutional layer, used for generating higher resolution images from lower resolution ones
    \item for increasing the resolution of images
    \item for reversing convolution operations
\end{itemize}

\subsection{ZeroPadding2D}
\begin{itemize}
    \item adds zero padding around the edges of input data
    \item allows for controlling the size of the output image after applying a convolutional layer
    \item for maintaining spatial size of input data after convolution
\end{itemize}

\subsection{LeakyReLU}
\begin{itemize}
    \item a variant of the ReLU activation function that allows for a small gradient when the unit is not activated (negative value)
    \item for preventing the "vanishing gradients" problem
\end{itemize}

\subsection{InstanceNormalization / BatchNormalization}
\begin{itemize}
    \item normalizes input data for each sample independently
    \item allows for stabilizing the training process and rapid convergence
    \item for normalizing input features in neural networks, especially in generative models and image stylization
\end{itemize}

\subsection{ResizeLayer}
\begin{itemize}
    \item resizes input data to a specified dimension
    \item for resizing images in neural networks (for standardizing input or output image sizes)
\end{itemize}

\end{document}
