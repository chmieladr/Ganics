\documentclass{article}
\usepackage[utf8]{inputenc}
\usepackage[T1]{fontenc}
\usepackage{geometry}
\usepackage{enumitem}
\usepackage{graphicx}

\geometry{margin=0.75in}

\begin{document}

\title{\textbf{Notatka o GAN-ach}
\\ \large{\textit{Ganics}}}
\author{\textbf{Adrian Chmiel}}
\date{4 czerwca 2024}
\maketitle

\section{Zwykły GAN}
\begin{itemize}
    \item jeden generator i jeden dyskryminator
    \item trening na sparowanych danych mających na celu generowanie realistycznych ogólnych obrazów
\end{itemize}

\section{CycleGAN}
\begin{itemize}
    \item dwa generatory i dwa dyskryminatory
    \item trening na niesparowanych danych mających na celu konwersję stylu obrazów
    \item \textbf{cykl konsystencji straty} - obraz przetłumaczony z A na B, a następnie z powrotem z B na A powinien być podobny do oryginalnego obrazu
\end{itemize}

\section{SRGAN}
\begin{itemize}
    \item jeden generator i jeden dyskryminator
    \item trening mający na celu generowanie wysokiej jakości obrazów
    \item \textbf{połączenia resztkowe} - przekazywanie informacji z poprzednich warstw do kolejnych z pominięciem warstw pośrednich
\end{itemize}

\section{Dyskryminator PatchGAN}
\begin{itemize}
    \item dyskryminator oceniający obrazy na poziomie patchy
    \item zamiast oceny całego obrazu, ocenia on poszczególne fragmenty obrazu
    \item pozwala na bardziej szczegółową ocenę obrazów
    \item często stosowany jako część \textit{SRGAN}-ów
\end{itemize}

\end{document}